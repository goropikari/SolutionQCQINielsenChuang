%
% generate PDF file
% pdflatex solnQCQI.tex
%
\documentclass[11pt]{book}
\usepackage[utf8]{inputenc}
\usepackage{amsmath,amsfonts,amssymb,amsthm}
\usepackage[width=16.00cm, height=24.00cm]{geometry}
\usepackage[dvipdfmx]{graphicx}
\usepackage[english]{babel}

%図の場所をなるべく指定した場所にする
\usepackage{booktabs}
\usepackage{here}

\RequirePackage[l2tabu, orthodox]{nag}
\usepackage[all, warning]{onlyamsmath}

%単位を書くときに使う
\usepackage{siunitx}

\usepackage{CJKutf8}
\usepackage{ascmac} % screen
\usepackage{ulem}
\usepackage{cases}
\usepackage{braket}
\usepackage{dsfont}
\usepackage{ascmac}
\usepackage{url}
\usepackage{hyperref} % hyper link
\usepackage{ccicons} % creative commons license icon
\usepackage{fancyhdr} % footer
%\pagestyle{fancy}
%\cfoot[\href{http://creativecommons.org/licenses/by-nc-sa/4.0/}{Creative Commons Attribution-NonCommercial-ShareAlike 4.0 International License}.]{}
\usepackage{color}


\usepackage{fancyhdr}
\setlength{\headheight}{15.2pt}
\pagestyle{fancy}
\lhead[\leftmark ]{\thepage}
\rhead[\thepage]{\leftmark}

\cfoot{\footnotesize \textcopyright 2018 goropikari - \href{http://creativecommons.org/licenses/by-nc-sa/4.0/}{Creative Commons Attribution-NonCommercial-ShareAlike 4.0 International License}}

% コマンド定義
\DeclareMathOperator{\Tr}{Tr}
\newcommand{\norm}[1]{\left\lVert#1\right\rVert} % norm ||x||
\newcommand{\kb}[1]{\ket{#1}\hspace{-1mm} \bra{#1}} % |x><x|
\newcommand{\kbt}[2]{\ket{#1}\hspace{-1mm} \bra{#2}} % |x><y|
\newcommand{\Textbf}[1]{\hspace{3mm}\\ \textbf{#1}\\}
\newtheorem{thm}{Theorem.}[section]
\newtheorem{prop}{Proposition.}[section]


\title{Solution for "Quantum Computation and Quantum Information: 10th Anniversary Edition" by Nielsen and Chuang}
\author{goropikari}
\date{\today}

\begin{document}
\maketitle
\thispagestyle{empty}
\setcounter{page}{0} % 表紙のページを0ページにする

\section*{Copylight Notice:}
\ccbyncsa\\
This work is licensed under a \href{http://creativecommons.org/licenses/by-nc-sa/4.0/}{Creative Commons Attribution-NonCommercial-ShareAlike 4.0 International License}.


\section*{Repository}
The latest version and source \LaTeX code are located in\\ \url{https://github.com/goropikari/SolutionForQuantumComputationAndQuantumInformation}.

\section*{For readers}
This is an unofficial solution manual for "\href{http://www.cambridge.org/jp/academic/subjects/physics/quantum-physics-quantum-information-and-quantum-computation/quantum-computation-and-quantum-information-10th-anniversary-edition?format=HB&isbn=9781107002173#BBFv83H3ofgcgG3A.97}{Quantum Computation and Quantum Information: 10th Anniversary Edition}" (ISBN-13: 978-1107002173) by Michael A. Nielsen and Isaac L. Chuang.



I have studied quantum information theory as a hobby.
And I'm not a researcher.
So there is no guarantee that these solutions are correct.
Especially because I'm not good at mathematics, proofs are often wrong.
Don't trust me. Verify yourself!

If you find some mistake or have some comments, please feel free to open an issue or a PR.
\begin{flushright}
    \href{https://github.com/goropikari}{goropikari}
\end{flushright}

\tableofcontents
\newpage

%%%%%%%%%%%%%%%%%%%%%%%%%%%%%%%%%%%%%%%%%%%%%%%%%%%%%%%%%%%%%%%%%%%%%%%%%%%%%
\frontmatter
\section{Errata list}\label{errata}

\begin{itemize}
    \item p.101. eq (2.150) $\rho = \sum_m p(m) \rho_m$ should be $\rho \textcolor{red}{'} = \sum_m p(m) \rho_m$.
%
    \item p.408. eq (9.49) $\sum_i p_i D(\rho_i, \sigma_i) + D(p_i, q_i)$ should be $\sum_i p_i D(\rho_i, \sigma_i) + \textcolor[rgb]{1,0,0}{2}D(p_i, q_i)$.

    \begin{align*}
    \text{eqn } (9.48) &= \sum_i p_i \Tr (P(\rho_i - \sigma_i)) + \sum_i (p_i - q_i) \Tr (P \sigma_i)\\
    &\leq \sum_i p_i \Tr (P(\rho_i - \sigma_i)) + \sum_i (p_i - q_i)~~~(\because \Tr (P \sigma_i) \leq 1)\\
    &= \sum_i p_i \Tr (P(\rho_i - \sigma_i)) + 2 \frac{\sum_i (p_i - q_i)}{2}\\
    &= \sum_i p_i \Tr (P(\rho_i - \sigma_i)) + 2D(p_i, q_i)
    \end{align*}
%
    \item p.409. Exercise 9.12. If $\rho = \sigma$, then $D(\rho, \sigma) = 0$. Furthermore trace distance is non-negative. Therefore $0 \leq D(\mathcal{E}(\rho), \mathcal{E}(\sigma)) \leq 0 \Rightarrow D(\mathcal{E}(\rho), \mathcal{E}(\sigma))  = 0$. So I think the map $\mathcal{E}$ is not strictly contractive. If $p \neq 1$ and $\rho \neq \sigma$, then $D(\mathcal{E}(\rho), \mathcal{E}(\sigma)) < D(\rho, \sigma)$ is satisfied.
%
    \item p.411. Exercise 9.16. eqn(9.73) $\Tr (A^\dagger B) = \braket{m | A \otimes B | m}$ should be $\Tr (A^{\textcolor{red}{T}} B) = \braket{m | A \otimes B | m}$.

    Simple counter example is the case that
    $A = \begin{bmatrix}
        i & 0\\
        0 & 0
    \end{bmatrix}$.
    $B = \begin{bmatrix}
        1 & 0\\
        0 & 0
    \end{bmatrix}$,
    In this case,
    \begin{align*}
        A^\dagger B &=
        \begin{bmatrix}
            -i & 0\\
            0 & 0
        \end{bmatrix}
        \begin{bmatrix}
            1 & 0\\
            0 & 0
        \end{bmatrix}
        = \begin{bmatrix}
            -i & 0\\
            0 & 0
        \end{bmatrix},\\
%
        \Tr (A^\dagger B)& = -i,\\
%
        A \otimes B &= \begin{bmatrix}
            i & 0 & 0 & 0\\
            0 & 0 & 0 & 0\\
            0 & 0 & 0 & 0\\
            0 & 0 & 0 & 0
        \end{bmatrix}\\
        \braket{m | A \otimes B | m} &= (\bra{00} + \braket{11}) (A \otimes B) (\ket{00} + \ket{11}) = i.
    \end{align*}
    Thus $\Tr(A^\dagger B) \neq \braket{m | A \otimes B | m}$.

    By using following relation, we can prove.
    \begin{align*}
        (I \otimes A) \ket{m} = (A^T \otimes I) \ket{m}\\
        \Tr (A) = \braket{m | I \otimes A | m}
    \end{align*}
%
    \begin{align*}
        \Tr (A^T B) = \Tr(BA^T) &= \braket{m | I \otimes BA^T |m}\\
            &= \braket{m | (I \otimes B)(I \otimes A^T) |m}\\
            &= \braket{m | (I \otimes B)(A \otimes I) |m}\\ 
            &= \braket{m | A \otimes B | m}.
    \end{align*}

    \item p.515. eqn (11.67) $S(\rho' || \rho)$ should be $S(\rho || \rho \textcolor{red}{'})$.
\end{itemize}


\mainmatter
\setcounter{section}{1}
\section{Introduction to quantum mechanics}
\subsection{}
\begin{align}
	\begin{bmatrix}
		1 \\ 
		-1
	\end{bmatrix} 
	+
	\begin{bmatrix}
		1 \\ 
		2
	\end{bmatrix}
	-
	\begin{bmatrix}
		2 \\ 
		1
	\end{bmatrix}
	=
	\begin{bmatrix}
		0 \\ 
		0
	\end{bmatrix}	
\end{align}

\subsection{}
\begin{align}
	A\ket{0} &= A_{11}\ket{0} + A_{21}\ket{1} = \ket{1} \Rightarrow A_{11} = 0,\ A_{21} = 1\\
	A\ket{1} &= A_{12}\ket{0} + A_{22}\ket{1} = \ket{0} \Rightarrow A_{12} = 1,\ A_{22} = 0\\
%
	\therefore A &= 	
	\begin{bmatrix}
		0 & 1 \\ 
		1 & 0
	\end{bmatrix} 
\end{align}\\

\begin{align}
	A &= 	
	\begin{bmatrix}
	1 & 0 \\ 
	0 & 1
	\end{bmatrix} \text{ w.r.t. } \left\{\ket{1},\ \ket{0} \right\}
\end{align}


\subsection{}
\begin{align}
	A \ket{v_i} &= \sum_{j} A_{ji}\ket{w_j}\\
	B \ket{w_j} &= \sum_{k} B_{kj}\ket{x_k}
\end{align}
%
Thus
\begin{align}
	BA \ket{v_i} &= B \left( \sum_{j} A_{ji}\ket{w_j} \right)\\
	&= \sum_{j} A_{ji} B\ket{w_j}\\
	&= \sum_{j,k} A_{ji} B_{kj}\ket{x_k}\\
	&= \sum_k \left( \sum_j B_{kj} A_{ji}  \right) \ket{x_k}\\
	&= \sum_k (BA)_{ki} \ket{x_k}
\end{align}



\subsection{}
\begin{align}
	I\ket{v_j} = \sum_i I_{ij} \ket{v_i} = \ket{v_j},\ \forall j.\\
	\Rightarrow I_{ij} = \delta_{ij}
\end{align}

\setcounter{subsection}{5}
\subsection{}
\begin{align}
	\left(\sum_i \lambda_i \ket{w_i},\ \ket{v}\right) &=
	\left(\ket{v},\ \sum_i \lambda_i \ket{w_i}\right)^*\\
	&= \left[\sum_i \lambda_i \left(\ket{v},\ \ket{w_i}  \right) \right]^*\\
	&= \sum_i \lambda_i^* \left(\ket{v},\ \ket{w_i} \right)^*\\
	&= \sum_i \lambda_i^* (\ket{w_i},\ \ket{v})
\end{align}


\subsection{}
\begin{align}
	\braket{w | v} &= \begin{bmatrix}
		1 & 1
	\end{bmatrix} 
	\begin{bmatrix}
	1 \\ 
	-1
	\end{bmatrix} 
	= 1 - 1 = 0\\
%	
	\frac{\ket{w}}{\norm{\ket{w}}} &= 
	\frac{\ket{w}}{\sqrt{\braket{w|w}}} = \frac{1}{\sqrt{2}} \begin{bmatrix}
	1 \\ 
	1
	\end{bmatrix}\\
%	
	\frac{\ket{v}}{\norm{\ket{v}}} &= 
	\frac{\ket{v}}{\sqrt{\braket{v|v}}} = \frac{1}{\sqrt{2}} \begin{bmatrix}
	1 \\ 
	-1
	\end{bmatrix}
\end{align}


\setcounter{subsection}{8}
\subsection{}
\begin{align}
	\sigma_0 &= I = \ket{0}\bra{0} + \ket{1}\bra{1}\\
	\sigma_1 &= X = \ket{0}\bra{1} + \ket{1}\bra{0}\\
	\sigma_2 &= Y = -i\ket{0}\bra{1} + i\ket{1}\bra{0}\\
	\sigma_3 &= Z = \ket{0}\bra{0} - \ket{1}\bra{1}
\end{align}


\subsection{}
\begin{align}
	\ket{v_j}\bra{v_k} &= I_V \ket{v_j} \bra{v_k} I_V\\
	&= \left(\sum_p \ket{v_p}\bra{v_p} \right) \ket{v_j}\bra{v_k} \left(\sum_q \ket{v_q}\bra{v_q} \right)\\
	&= \sum_{p,q} \ket{v_p} \braket{v_p|v_j}
	\braket{v_k | v_q} \bra{v_q}\\
	&= \sum_{p,q} \delta_{pj} \delta_{kq} \ket{v_p} \bra{v_q}
\end{align}
Thus
\begin{align}
	\left( \ket{v_j}\bra{v_k} \right)_{pq} = \delta_{pj} \delta_{kq}
\end{align}



\subsection{}
\begin{align}
	X = \begin{bmatrix}
	0 & 1 \\ 
	1 & 0
	\end{bmatrix},\ \det(X-\lambda I) = 
	\det \left(\begin{bmatrix}
	-\lambda & 1 \\ 
	1 & -\lambda
	\end{bmatrix} \right) = 0 \Rightarrow \lambda \pm 1
\end{align}

If $\lambda = -1$,
\begin{align}
	\begin{bmatrix}
		1 & 1 \\ 
		1 & 1
	\end{bmatrix} 
	\begin{bmatrix}
		c_1 \\ 
		c_2 
	\end{bmatrix} = 
	\begin{bmatrix}
		0 \\ 
		0 
	\end{bmatrix}
\end{align}
Thus
\begin{align}
	\ket{\lambda = -1} = \begin{bmatrix}
	c_1 \\ 
	c_2 
	\end{bmatrix} = \frac{1}{\sqrt{2}} 
	\begin{bmatrix}
	-1 \\ 
	1 
	\end{bmatrix}
\end{align}

If $\lambda = 1$
\begin{align}
	\ket{\lambda = 1} = \frac{1}{\sqrt{2}} 
	\begin{bmatrix}
	1 \\ 
	1 
	\end{bmatrix}
\end{align}

\begin{align}
	X = \begin{bmatrix}
	-1 & 0 \\ 
	0 & 1
	\end{bmatrix} 
	\text{ w.r.t. } \left\{ \ket{\lambda = -1},\ \ket{\lambda = 1}\right\}
\end{align}



\subsection{}
\begin{align}
	\det \left(\begin{bmatrix}
	1 & 0 \\ 
	1 & 1
	\end{bmatrix} - \lambda I \right) = (1 - \lambda)^2 = 0 \Rightarrow \lambda = 1
\end{align}
Therefore the eigenvector associated with eigenvalue $\lambda = 1$ is 
\begin{align}
	\ket{\lambda = 1} = \begin{bmatrix}
	0 \\ 
	1
	\end{bmatrix} 
\end{align}

Because $\ket{\lambda = 1}\bra{\lambda = 1} = \begin{bmatrix}
0 & 0 \\ 
0 & 1
\end{bmatrix}$, 
\begin{align}
	\begin{bmatrix}
	1 & 0 \\ 
	1 & 1
	\end{bmatrix} \neq c\ket{\lambda = 1}\bra{\lambda = 1} = \begin{bmatrix}
	0 & 0 \\ 
	0 & c
	\end{bmatrix}
\end{align}



\subsection{}
Suppose $\ket{\psi},\ \ket{\phi}$ are arbitrary vectors in $V$.
\begin{align}
	\left(\ket{\psi},\ (\ket{w}\bra{v}) \ket{\phi}\right)^* &=
	\left((\ket{w}\bra{v})^\dagger \ket{\psi},\  \ket{\phi}\right)^*\\
	&= \left(\ket{\phi},\ (\ket{w}\bra{v})^\dagger \ket{\psi} \right)\\
	&= \bra{\phi} (\ket{w}\bra{v})^\dagger \ket{\psi}.
\end{align}

On the other hand,
\begin{align}
	\left(\ket{\psi},\ (\ket{w}\bra{v}) \ket{\phi}\right)^* 
	&= (\braket{\psi | w} \braket{v | \phi})^*\\
	&= \braket{\phi | v} \braket{w | \psi}.
\end{align}

Thus
\begin{align}
	\bra{\phi} (\ket{w}\bra{v})^\dagger \ket{\psi} = \braket{\phi | v} \braket{w | \psi} \text{ for arbitrary vectors } \ket{\psi},\ \ket{\phi}\\
	\therefore (\ket{w}\bra{v})^\dagger = \ket{v}\bra{w}
\end{align}


\subsection{}
\begin{align}
	( (a_i A_i)^\dagger \ket{\phi},\ \ket{\psi} )
	&= (\ket{\phi},\ a_i A_i \ket{\psi})\\
	&= a_i (\ket{\phi},\ A_i \ket{\psi})\\
	&= a_i (A_i^\dagger \ket{\phi},\ \ket{\psi})\\
	&= (a_i^* A_i^\dagger \ket{\phi},\ \ket{\psi})\\
%	
	\therefore (a_i A_i)^\dagger = a_i^* A_i^\dagger
\end{align}




\subsection{}
\begin{align}
	((A^\dagger)^\dagger\ket{\psi},\ \ket{\phi} )
	&= (\ket{\psi},\ A^\dagger \ket{\phi})\\
	&= (A^\dagger \ket{\phi},\ \ket{\psi})^*\\
	&= (\ket{\phi},\ A\ket{\psi})^*\\
	&= (A\ket{\psi},\ \ket{\phi})\\
	\therefore (A^\dagger)^\dagger = A
\end{align}


\subsection{}
\begin{align}
	P &= \sum_i \ket{i}\bra{i}.\\
	P^2 &= \left(\sum_i \ket{i}\bra{i}\right) \left(\sum_j \ket{j}\bra{j}\right)\\
	&= \sum_{i,j} \ket{i}\braket{i | j}\bra{j}\\
	&= \sum_i \ket{i}\bra{j} \delta_{ij}\\
	&= \sum_i \ket{i}\bra{i}\\
	&= P
\end{align}




\setcounter{subsection}{17}
\subsection{}
Suppose $\ket{v}$ is a eigenvector with corresponding eigenvalue $\lambda$.
\begin{align}
	U \ket{v} &= \lambda \ket{v}.\\
	1 &= \braket{v | v}\\
	&= \bra{v} I \ket{v}\\
	&= \bra{v} U^\dagger U \ket{v}\\
	&= \lambda \lambda^* \braket{v | v}\\
	&= \norm{\lambda}^2\\
	\therefore \lambda &= e^{i \theta}
\end{align}




\subsection{}
\begin{align}
	X^2 = \begin{bmatrix}
		0 & 1 \\ 
		1 & 0
	\end{bmatrix} 
	\begin{bmatrix}
		0 & 1 \\ 
		1 & 0
	\end{bmatrix}
	= \begin{bmatrix}
		1 & 0 \\ 
		0 & 1
	\end{bmatrix} = I
\end{align}



\subsection{}
\begin{align}
	U &\equiv \sum_i \ket{w_i}\bra{v_i}\\
	A_{ij}^{'} &= \braket{v_i | A | v_j}\\
	&= \braket{v_i | UU^\dagger A UU^\dagger | v_j}\\
	&= \sum_{p,q,r,s} \braket{v_i | w_p} \braket{v_p | v_q} \braket{w_q | A | w_r} \braket{v_r | v_s} \braket{w_s | v_j}\\
	&= \sum_{p,q,r,s} \braket{v_i | w_p} \delta_{pq} A_{qr}^{''} \delta_{rs}  \braket{w_s | v_j}\\
	&= \sum_{p,r}  \braket{v_i | w_p}  \braket{w_r | v_j} A_{pr}^{''}
\end{align}


\setcounter{subsection}{25}
\subsection{}






\setcounter{chapter}{7}
\setcounter{chapter}{7}
\chapter{Quantum noise and quantum operations}
\Textbf{8.1}
\Textbf{8.2}
\Textbf{8.3}
\Textbf{8.4}
\Textbf{8.5}
\Textbf{8.6}
\Textbf{8.7}
\Textbf{8.8}
\Textbf{8.9}
\Textbf{8.10}
\Textbf{8.11}
\Textbf{8.12}
\Textbf{8.13}
\Textbf{8.14}
\Textbf{8.15}
\Textbf{8.16}
\Textbf{8.17}
\Textbf{8.18}
\Textbf{8.19}
\Textbf{8.20}
\Textbf{8.21}
\Textbf{8.22}
\Textbf{8.23}
\Textbf{8.24}
\Textbf{8.25}
\Textbf{8.26}
\Textbf{8.27}
\Textbf{8.28}
\Textbf{8.29}
\Textbf{8.30}
\Textbf{8.31}
\Textbf{8.32}
\Textbf{8.33}
\Textbf{8.34}
\Textbf{8.35}

\chapter{Distance measures for quantum information}
\Textbf{9.1}
\begin{align*}
	D((1,0), (1/2,1/2)) &= \frac{1}{2} \left( |1 - 1/2| + |0 - 1/2| \right)\\
		&= \frac{1}{2} \left(\frac{1}{2} +  \frac{1}{2} \right)\\
		&= \frac{1}{2}
\end{align*}

\begin{align*}
	D\left( (1/2,1/3,1/6), (3/4, 1/8, 1/8)  \right) &= \frac{1}{2} \left( |1/2 - 3/4| + |1/3 - 1/8| + |1/6 - 1/8| \right)\\
		&= \frac{1}{2} \left( 1/4 + 5 / 24 + 1/24 \right)\\
		&= \frac{1}{4}
\end{align*}



\Textbf{9.2}
\begin{align*}	
	D\left((p,1-p), (q, 1-q)\right) &= \frac{1}{2} \left( |p-q| + |(1-p) - (1-q)| \right)\\
		&= \frac{1}{2} \left( |p-q| + |-p + q| \right)\\
		&= |p-q| 
\end{align*}


\Textbf{9.3}
\begin{align*}
	F((1,0), (1/2,1/2)) = \sqrt{1 \cdot 1/2} + \sqrt{0 \cdot 1/2} = \frac{1}{\sqrt{2}}
\end{align*}

\begin{align*}
	F\left( (1/2,1/3,1/6), (3/4, 1/8, 1/8)  \right) &= \sqrt{1/2 \cdot 3/4} + \sqrt{1/3 \cdot 1/8} + \sqrt{1/6 \cdot 1/8}\\
		&= \frac{4 \sqrt{6} + \sqrt{3}}{12}
\end{align*}



\Textbf{9.4}

Define $r_x = p_x - q_x$. Let $U$ be the whole index set.
\begin{align*}
	\max_S |p(S) - q(S)| &= \max_S \left|\sum_{x \in S} p_x - \sum_{x \in S} q_x \right|\\
		&= \max_S \left|\sum_{x \in S} (p_x -  q_x) \right|\\
		&= \max_S \left|\sum_{x \in S} r_x \right|
\end{align*}


Since $\sum_{x \in S} r_x $ is written as
\begin{align}
	\sum_{x \in S} r_x = \sum_{\substack{x \in S\\  r_x \geq 0}} r_x + \sum_{\substack{x \in S\\ r_x < 0}} r_x,\label{eq:subsettracedist}
\end{align}
$\left|\sum_{x \in S} r_x \right|$ is maximized when $S = \{x \in U | r_x \geq 0  \}$ or $S = \{x \in U | r_x < 0  \}$.

Define $S_+ = \{x \in U | r_x \geq 0  \}$ and $S_- = \{x \in U | r_x < 0  \}$.

Now the sum of all $r_x$ is 0,
\begin{align*}
	&\sum_{x \in U} r_x = \sum_{x \in S_+} r_x + \sum_{x \in S_-} r_x = 0\\
	\therefore &\sum_{x \in S_+} r_x = - \sum_{x \in S_-} r_x.
\end{align*} 

Thus
\begin{align}
	\max_S \left|\sum_{x \in S} r_x \right| = \sum_{x \in S_+} r_x = - \sum_{x \in S_-} r_x \label{eq:probtracedist}.
\end{align}

On the other hand,
\begin{align}
	D(p_x, q_x) &= \frac{1}{2} \sum_{x \in U} |p_x - q_x|\nonumber\\
		&= \frac{1}{2} \sum_{x \in U} |r_x|\nonumber\\
		&= \frac{1}{2} \sum_{x \in S_+} |r_x| + \frac{1}{2} \sum_{x \in S_-} |r_x|\nonumber\\
		&= \frac{1}{2} \sum_{x \in S_+} r_x - \frac{1}{2} \sum_{x \in S_-} r_x\nonumber\\
		&= \frac{1}{2} \sum_{x \in S_+} r_x + \frac{1}{2} \sum_{x \in S_+} r_x ~~~(\because \text{eqn} (\ref{eq:probtracedist}))\nonumber\\
		&= \sum_{x \in S_+} r_x \nonumber\\
		&= \max_S \left|\sum_{x \in S} r_x \right|.\nonumber
\end{align}

Therefore $D(p_x, q_x) = \max_S \left|\sum_{x \in S} p_x - \sum_{x \in S}q_x \right| = \max_S |p(S) - q(S)|$.

\Textbf{9.5}
From eqn (\ref{eq:subsettracedist}) and (\ref{eq:probtracedist}), maximizing $\left|\sum_{x \in S} r_x \right|$ is equivalent to 
maximizing $\sum_{x \in S} r_x$.

Hence
\begin{align*}
	D(p_x, q_x) = \max_S (p(S) - q(S)) = \max_S \left(\sum_{x \in S} p_x - \sum_{x \in S}q_x \right).
\end{align*}



\Textbf{9.6}

Define $\rho = \frac{3}{4}\kb{0} + \frac{1}{4}\kb{1}$, $\sigma = \frac{2}{3} \kb{1} + \frac{1}{3}\kb{1}$.
\begin{align*}
	D(\rho, \sigma) &= \frac{1}{2} \Tr |\rho - \sigma|\\
		&= D((3/4, 1/4), (2/3, 1/3))\\
		&= \frac{1}{2} \left(\left| \frac{3}{4} - \frac{2}{3}  \right| +\left| \frac{1}{4} - \frac{1}{3} \right| \right)\\
		&= \frac{1}{2} \left( \frac{1}{12} + \frac{1}{12} \right)\\
		&= \frac{1}{12}
\end{align*}


Define $\rho = \frac{3}{4}\kb{0} + \frac{1}{4}\kb{1}$, $\sigma = \frac{2}{3} \kb{+} + \frac{1}{3}\kb{-}$.
\begin{align*}
	\kb{+} &= \frac{1}{2} (\kb{0} + \kbt{0}{1} + \kbt{1}{0} + \kb{1})\\
	\kb{-} &= \frac{1}{2} (\kb{0} - \kbt{0}{1} - \kbt{1}{0} + \kb{1})
\end{align*}

\begin{align*}
	\rho - \sigma &= \left(\frac{3}{4} - \frac{1}{2}\right) \kb{0} - \frac{1}{6} (\kbt{0}{1} + \kbt{1}{0}) + \left(\frac{1}{4} - \frac{1}{2}\right) \kb{1}\\
		&= \frac{1}{4} \kb{0} - \frac{1}{6} (\kbt{0}{1} + \kbt{1}{0}) - \frac{1}{4} \kb{1}\\
\end{align*}

\begin{align*}
	(\rho - \sigma)^\dagger (\rho - \sigma) &= \frac{1}{4^2} \kb{0} - \frac{1}{4\cdot 6} \kbt{0}{1} + \frac{1}{6^2} \kb{0} + \frac{1}{6 \cdot 4} \kbt{0}{1} - \frac{1}{4\cdot 6} \kbt{1}{0} + \frac{1}{6^2} \kb{1} + \frac{1}{4\cdot 6} \kbt{1}{0} + \frac{1}{4^2} \kb{1}\\
		&= \left(\frac{1}{4^2} + \frac{1}{6^2}\right) (\kb{0} + \kb{1})
\end{align*}


\begin{align*}
	D(\rho, \sigma) &= \frac{1}{2} \Tr |\rho - \sigma|\\
		&= \sqrt{\frac{1}{4^2} + \frac{1}{6^2}}
\end{align*}



\Textbf{9.7}

Since $\rho - \sigma$ is Hermitian, we can apply spectral decomposition.
Then $\rho - \sigma$ is written as
\begin{align*}
	\rho - \sigma &= \sum_{i=1}^k \lambda_i \kb{i} + \sum_{i=k+1}^{n} \lambda_i \kb{i}
\end{align*}
where $\lambda_i$ are  positive eigenvalues for $i = 1, \cdots, k$ and negative eigenvalues for $i = k+1, \cdots, n$.

Define $Q = \sum_{i=1}^k \lambda_i \kb{i}$ and $S = -\sum_{i=k+1}^{n} \lambda_i \kb{i}$.
Then $P$ and $S$ are positive operator. Therefore $\rho - \sigma = P - S$.

\begin{screen}
	Proof of $|\rho - \sigma| = Q + S$.
	\begin{align*}
		|\rho- \sigma| &= |Q - S|\\
			&= \sqrt{(Q-S)^\dagger (Q-S)}\\
			&= \sqrt{(Q-S)^2}\\
			&= \sqrt{Q^2 - QS - SQ + S^2}\\
			&= \sqrt{Q^2 + S^2}\\
			&= \sqrt{\sum_i \lambda_i^2 \kb{i}}\\
			&= \sum_i |\lambda_i| \kb{i}\\
			&= Q + S
	\end{align*}
\end{screen}

\Textbf{9.8}

Suppose $\sigma = \sigma_i$. Then $\sigma = \sum_i p_i \sigma_i$.
\begin{align}
	D \left( \sum_i p_i \rho_i, \sigma\right) &= D \left( \sum_i p_i \rho_i, \sum_i p_i \sigma_i\right)\\
		&\leq \sum_i p_i D(\rho_i, \sigma_i) ~~~ (\because \text{eqn}(9.50))\\
		&= \sum_i p_i D(\rho_i, \sigma).~~(\because \text{assumption}).
\end{align}

\Textbf{9.9}
\Textbf{9.10}
\Textbf{9.11}
\Textbf{9.12}

Suppose $\rho = \frac{1}{2} (I + \vec{r}\cdot \vec{\sigma})$ and $\sigma = \frac{1}{2} (I + \vec{s}\cdot \vec{\sigma})$ where $\vec{v}$ and $\vec{s}$ are real vectors s.t. $|\vec{v}|, |\vec{s}| \leq 1$.

\begin{align*}
	\mathcal{E} (\rho) = p \frac{I}{2} + (1-p) \rho, ~~~
	\mathcal{E}(\sigma) = p \frac{I}{2} + (1-p) \sigma.
\end{align*}

\begin{align*}
	D(\mathcal{E}(\rho), \mathcal{E}(\sigma)) &= \frac{1}{2} \Tr |\mathcal{E}(\rho) -  \mathcal{E}(\sigma)|\\
		&= \frac{1}{2} \Tr |(1-p)(\rho - \sigma)|\\
		&= \frac{1}{2}(1-p) \Tr |\rho - \sigma|\\
		&= (1-p) D(\rho, \sigma)\\
		&= (1-p) \frac{|\vec{r} - \vec{s}|}{2}
\end{align*}

Is this strictly contractive?

\Textbf{9.13}

Bit flip channel $E_0 = \sqrt{p} I$,  $E_1 = \sqrt{1-p}\sigma_x$.
\begin{align*}
	\mathcal{E}(\rho) &= E_0 \rho E_0^\dagger + E_1 \rho E_1^\dagger\\
		&= p \rho + (1-p) \sigma_x \rho \sigma_x.
\end{align*}

Since $\sigma_x \sigma_x \sigma_x = \sigma_x$, $\sigma_x \sigma_y \sigma_x = -\sigma_y$ and $\sigma_x \sigma_z \sigma_x = -\sigma_z$, then $\sigma_x (\vec{r} \cdot \vec{\sigma}) = r_1 \sigma_x - r_2 \sigma_y - r_3 \sigma_3$.

Thus
\begin{align*}
	D(\mathcal{E}(\rho), \mathcal{E}(\sigma)) &= \frac{1}{2} \Tr |\mathcal{E}(\rho) -  \mathcal{E}(\sigma)|\\
		&= \frac{1}{2} \Tr |p(\rho - \sigma) + (1-p) (\sigma_x \rho \sigma_x - \sigma_x \sigma \sigma_x) |\\
		&\leq \frac{1}{2}p \Tr |\rho - \sigma | + \frac{1}{2} (1-p) \Tr |\sigma_x (\rho - \sigma) \sigma_x|\\
		&= pD(\rho, \sigma) +  (1-p) D(\sigma_x \rho \sigma_x, \sigma_x \sigma \sigma_x)\\
		&= D(\rho, \sigma)~~~(\because \text{eqn}(9.21)).
\end{align*}


Suppose $\rho_0 = \frac{1}{2}(I + \vec{r}\cdot \vec{\sigma})$ is a fixed point. Then
\begin{align*}
	&\rho_0 = \mathcal{E}(\rho_0) =p \rho_0 + (1-p) \sigma_x \rho_0 \sigma_x\\
	\therefore~ &(1-p) \rho_0 - (1-p) \sigma_x \rho_0 \sigma_x = 0\\
	\therefore~ &(1-p) (\rho - \sigma_x \rho_0 \sigma_x) = 0\\
	\therefore~ &\rho_0 = \sigma_x \rho_0 \sigma_x\\
	\therefore~ &\frac{1}{2} (I + r_1 \sigma_x + r_2 \sigma_y + r_3 \sigma_z)  \frac{1}{2} (I + r_1 \sigma_x - r_2 \sigma_y - r_3 \sigma_z)
\end{align*}

Since $\{I, \sigma_x, \sigma_y, \sigma_z \}$ are linearly independent, $r_2 = -r_2$ and $r_3 = - r_3$. Thus $r_2 = r_3 = 0$.

Therefore the set of fixed points for the bit flip channel is $\{\rho~ |~ \rho = \frac{1}{2}(I + r \sigma_x), |r| \leq 1, r \in \mathds{R} \}$


\Textbf{9.14}

\begin{align*}
	F(U \rho U^\dagger, U \sigma U^\dagger) &= \Tr \sqrt{(U\rho U^\dagger)^{1/2} \sigma (U \rho U^\dagger)}\\
		&= \Tr \sqrt{U \rho^{1/2} \sigma \rho^{1/2} U^\dagger}\\
		&= \Tr (U \sqrt{\rho^{1/2} \sigma \rho^{1/2}} U^\dagger)\\
		&= \Tr (\sqrt{\rho^{1/2} \sigma \rho^{1/2}} U^\dagger U)\\
		&= \Tr \sqrt{\rho^{1/2} \sigma \rho^{1/2}}\\
		&= F(\rho, \sigma)
\end{align*}

\begin{screen}
	I think the fact $\sqrt{UAU^\dagger} = U\sqrt{A}U^\dagger$ is not restricted for positive operator.
	
	Suppose $A$ is a normal matrix. From spectral theorem, it is decomposed as
	\begin{align*}
		A = \sum_i a_i \kb{i}.
	\end{align*}
	
	Let $f$ be a function. Then
	\begin{align*}
		f(UAU^\dagger )&= f(\sum_i a_i U \kb{i} U^\dagger)\\
			&= \sum_i f(a_i) U \kb{i} U^\dagger\\
			&= U (\sum_i f(a_i) U \kb{i} U^\dagger) U^\dagger\\
			&= U f(A) U^\dagger
	\end{align*}
\end{screen}


\Textbf{9.15}
\Textbf{9.16}
\Textbf{9.17}
\Textbf{9.18}
\Textbf{9.19}
\Textbf{9.20}
\Textbf{9.21}
\Textbf{9.22}
\Textbf{9.23}
\setcounter{chapter}{10}
\chapter{Entropy and information}
\Textbf{11.1}
Fair coin:

\begin{align}
    H({1/2, 1/2}) = \left( - \frac{1}{2} \log \frac{1}{2} \right) \times 2 = 1
\end{align}


Fair die:
\begin{align}
    H(p) = \left( - \frac{1}{6} \log \frac{1}{6} \right) \times 6 = \log 6.
\end{align}


The entropy decreases if the coin or die is unfair.



\Textbf{11.2}

From assumption $I(pq) = I(p) + I(q)$.

\begin{align}
    \frac{\partial I(pq)}{\partial p} &= \frac{\partial I(p)}{\partial p} + 0 = \frac{\partial I(p)}{\partial p}\\
    \frac{\partial I(pq)}{\partial q} &= 0 + \frac{\partial I(q)}{\partial q} = \frac{\partial I(q)}{\partial q}
\end{align}


\begin{align}
    \frac{\partial I(pq)}{\partial p}
        &=  \frac{\partial I(pq)}{\partial (pq)} \frac{\partial (pq)}{\partial p}
        = q \frac{\partial I(pq)}{\partial(pq)}
    \Rightarrow \frac{\partial I(pq)}{\partial(pq)} = \frac{1}{q} \frac{\partial I(p)}{\partial p}\\
%
    \frac{\partial I(pq)}{\partial q}
        &=  \frac{\partial I(pq)}{\partial (pq)} \frac{\partial (pq)}{\partial q}
        = p \frac{\partial I(pq)}{\partial(pq)}
    \Rightarrow \frac{\partial I(pq)}{\partial(pq)} = \frac{1}{p} \frac{\partial I(q)}{\partial q}
\end{align}

Thus
\begin{align}
    \frac{1}{q} \frac{\partial I(p)}{\partial p} &= \frac{1}{p} \frac{\partial I(q)}{\partial q}\\
    \therefore~ p \frac{d I(p)}{d p} &= q \frac{d I(q)}{d q} ~~~\text{ for all } p,q \in [0,1].\\
\end{align}

Then $p (d I(p) / d p)$ is constant.

If $p (d I(p) / d p) = k$, $k \in \mathds{R}$.
Then $I(p) = k \ln p = k' \log p$ where $k' = k / \log e$.



\Textbf{11.3}
$H_{\text{bin}}(p) = - p\log p - (1-p) \log (1-p)$.

\begin{align}
    \frac{d H_{\text{bin}}(p)}{d p}
        &= \frac{1}{\ln 2} \left( - \log p - 1 + \log (1-p) + 1 \right)\\
        &= \frac{1}{\ln 2} \ln \frac{1-p}{p} = 0\\
    \Rightarrow \frac{1-p}{p} = 1\\
    \Rightarrow p = 1/2.
\end{align}


\Textbf{11.4}
\Textbf{11.5}
\Textbf{11.6}
\Textbf{11.7}
\Textbf{11.8}
\Textbf{11.9}
\Textbf{11.10}
\Textbf{11.11}
\Textbf{11.12}
\Textbf{11.13}
\Textbf{11.14}
\Textbf{11.15}
\Textbf{11.16}
\Textbf{11.17}
\Textbf{11.18}
\Textbf{11.19}
\Textbf{11.20}
\Textbf{11.21}
\Textbf{11.22}
\Textbf{11.23}
\Textbf{11.24}
\Textbf{11.25}
\Textbf{11.26}

\Textbf{Problem 11.1}
\Textbf{Problem 11.2}
\Textbf{Problem 11.3}
\Textbf{Problem 11.4}
\Textbf{Problem 11.5}



%%%%%%%%%%%%%%%%%%%%%%%%%%%%%%%%%%%%%%%%%%%%%%%%%%%%%%%%%%%%%%%%%%%%%%%%%%%%%

%参考文献
%\bibliographystyle{jplain}
%\bibliography{ref} % ref.bib を読み込み
\end{document}
